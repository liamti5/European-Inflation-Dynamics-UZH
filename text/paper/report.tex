\documentclass{article}
\usepackage{amsmath}
\usepackage{blindtext}
\usepackage{hyperref}
\usepackage{booktabs}
% bibliography
\usepackage[
    backend=biber,
    style=bwl-FU,
    url=false,
    doi=false,
    eprint=false
]{biblatex}
\addbibresource{citations.bib}

\title{Analysing the Effect of Longitude and Latitude on European Inflation Dynamics}
\author{Valentin Leuthard, Panagiotis Patsias, Liam Kane, Liam Tessendorf}


\begin{document}
    \maketitle
    \vspace{\baselineskip}
    \vspace{\baselineskip}    
    \vspace{\baselineskip}
    \begin{abstract}
    This project analyzes European inflation dynamics with a focus on the effects of longitude and latitude on inflation rates. By examining spatial patterns across different countries, the study aims to uncover how geographic location influences economic indicators within Europe. Utilizing data visualization and statistical modeling, the project provides insights into regional inflation trends and their relationship with geographical coordinates.
    \end{abstract}
    \newpage

    \tableofcontents
    \newpage

    \section{Introduction}

    Although inflation can be used as a financing tool to stimulate economic growth by incentivising people to purchase goods and invest their money rather than saving it, its benefits are limited. Rising prices erode the value of money and reduce people's purchasing power, negatively affecting especially those on fixed wages or pensions. Additionally, inflation can increase financial inequality by disproportionately harming wage earners more than asset owners (\cite{TheEffectofInflationonEconomicDevelopment}). These varied effects of inflation highlight the importance of understanding its underlying drivers.

    Our research focuses solely on the potential effect of the longitude and latitude of European countries on their inflation rates. Specifically, we aim to investigate whether these metrics have a measurable effect on the inflation in European countries. This approach is interesting, as the geographic location of a country inherently shapes its dynamics within the broader inter-country financial ecosystem, which we therefore hypothesise might influence its inflation.

    \section{Data}
    
    For our analysis we used data on European Inflation rates from the \href{https://data-explorer.oecd.org/vis?tm=inflation&pg=0&snb=50&vw=tb&df%5Bds%5D=dsDisseminateFinalDMZ&df%5Bid%5D=DSD_PRICES%40DF_PRICES_HICP&df%5Bag%5D=OECD.SDD.TPS&df%5Bvs%5D=1.0&dq=HRV%2BBGR%2BTUR%2BGBR%2BCHE%2BSVN%2BSWE%2BESP%2BSVK%2BPRT%2BPOL%2BNOR%2BNLD%2BLUX%2BLTU%2BLVA%2BIRL%2BITA%2BISL%2BHUN%2BGRC%2BDEU%2BFRA%2BFIN%2BEST%2BDNK%2BBEL%2BCZE%2BAUT.M.HICP.CPI.PA._T.N.GY&to%5BTIME_PERIOD%5D=false&pd=2000-01%2C2024-10}{Organisation for Economic Co-operation and Development}. It contains Inflation data of 29 European countries. For most countries, there is data from 2000 until September 2024, but for some, such as Switzerland, there is less. For this reason, we removed the missing timeperiods for all countries which results in a dataset of inflation rates from December 2005 until September 2024. For more insights into this data source, please see the notebook \emph{0.02-lte-oced-european-inflation-rates-1.ipynb}. 

    For the longitude and latitude data, we used Google's \href{https://developers.google.com/public-data/docs/canonical/countries_csv}{countries.csv}. It provides longitude and latitude data for all countries. While processing this dataset we remove all countries that are not in europe, and thus not of interest.

    \section{Methodology}

    Countries' shapes and sizes vary widely, which introduces some complexity when it comes to choosing how to represent the longitude and latitude of each country. After some deliberation, we opted for the geometric centres, also called centroids. Although Google's \href{https://developers.google.com/public-data/docs/canonical/countries_csv}{countries.csv} doesn't explicitly specify how the longitudes and latitudes are calculated, upon maunal inspection it became clear that their provided coordinates align closely with the countries' polygon centroids, making this dataset a suitable choice.
    
    Since inflation is a continuous metric, we employed linear regression as our model. Specifically, we performed regressions using longitude, latitude, as well as both variables combined. This allows us to analyze not only the effect of the absolute geographic positioning of a country on its inflation, but also conduct a differenciated analysis of its longitude and latitude individually.

    \section{Results}
    
    As mentioned in the last chapter we ran regressions on latitude, longitude as well as a combined regression of the inflation data on latitude and longitude. 

    \subsection{Regression on latitude}

    The regression using the latitude data of European countries yields the results in the table below.
   
    \vspace{\baselineskip}

    \begin{center}
    \begin{tabular}{cccc}
        \toprule
        {} &  Coefficient &  Std. Error &   P-Value \\
        \midrule
        Intercept &     7.108 &    4.0227 &  0.0885 \\
        latitude  &    -0.0767 &    0.0791 &  0.3413 \\
        \hline
        R-squared     &     0.043\\
        F-Statistic probability &   0.341\\
        \bottomrule
        
        \end{tabular}
    \end{center}
    \vspace{\baselineskip}

    These results are not statistically significant, since the P-value is 0.34 exceeding the normal margin for statistical significance for which it is 0.05.

    \subsection{Regression on longitude}

    Regression using the longitude data of the European countries yields the results in the table below.

    \vspace{\baselineskip}

    \begin{center}
    \begin{tabular}{cccc}
        \toprule
        {} &  Coefficient &  Std. Error &   P-Value \\
        \midrule
        Intercept &     1.9919 &    0.7193 &  0.0100 \\
        longitude &     0.1061 &    0.0424 &  0.0187 \\
        \hline
        R-squared     &     0.188\\
        F-Statistic probability &   0.0188\\
        \bottomrule
        \end{tabular}
    \end{center}
    \vspace{\baselineskip}

    This regression yields a statistically significant result with a P-value of 0.0424. This indicates a positive correlation between the latitude of a country and its mean inflation in the years between 2005 and 2024.

    \subsection{Combined regression on latitude \& longitude}

    The results of the multivariate regression using latitude and longitude data are depicted below.

    \vspace{\baselineskip}

    \begin{center}
    \begin{tabular}{cccc}
        \toprule
        {} &  Coefficient &  Std. Error &   P-Value \\
        \midrule
        Intercept &     4.8368 &    3.8383 &  0.2188 \\
        latitude  &    -0.0556 &    0.0737 &  0.4571 \\
        longitude &     0.1022 &    0.0430 &  0.0253 \\
        \hline
        R-squared     &    0.205\\
        F-Statistic Probability &   0.0503\\
        \bottomrule
        \end{tabular}
    \end{center}
    \vspace{\baselineskip}
    
    The results of the multivariate regression show a similar picture to the results of the univariate regressions. As latitude is still statistically not statistically significant, longitude remains statistically significant when using a statistical significance level of 0.05. Looking more closely at the values we can see that in the combined regression we have smaller coefficients and higher P-values than in the univariate regressions.
    
    \section{Discussion}
    This regression analysis investigates the relationship between inflation rates (dependent variable) and geographical coordinates (latitude and longitude) for European countries. The model is based on Ordinary Least Squares (OLS) regression.
    
    \subsection{Key Findings}

    The multivariate regression has a R-squared value of 0.205 meaning the regression longitude and latitude combined explain 20.5\% of the variation. This indicates a limited explanatory power of the regression.

    Latitude is statistically not significant. This indicates that there is no significant north-south effect of inflation.

    Longitude is statistically significant. This suggests that the inflation rates increase looking at countries more to the east. This seems to be influenced greatly by Turkey lies far to the east and has a very high inflation rate.

    \subsection{Implications}
    Longitude's significance suggests potential east-west economic trends, while latitude appears to have no impact. This is further supported by the correlation matrix, where longitude consistently shows a stronger relationship with inflation rates across all years (except 2009). However, the low R-squared indicates that geographic data alone is insufficient to model inflation, emphasizing the need to incorporate other economic factors for a more comprehensive analysis.

    \section{Conclusion}
    Geographical coordinates play a minor role in explaining inflation rates, with longitude showing some regional significance. Future models should integrate additional variables for better insight.

    \newpage
    \printbibliography

\end{document}