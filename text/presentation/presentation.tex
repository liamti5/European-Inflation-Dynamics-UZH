\documentclass[10pt]{beamer}

\usepackage{amsmath}
\usepackage{hyperref}
\usepackage[
    backend=biber,
    style=bwl-FU,
    url=false,
    doi=false,
    eprint=false
]{biblatex}
\addbibresource{citations.bib}

\title{Analysing the Effect of Longitude and Latitude on European Inflation Dynamics}
\author{Valentin Leuthard, Panagiotis Patsias, Liam Kane, Liam Tessendorf}

\begin{document}

% Title Page
\begin{frame}
  \titlepage
\end{frame}

% Introduction
\begin{frame}{Motivation \& Aim}
  \begin{itemize}
    \item Inflation can stimulate economic growth but has limitations. \footnote{\cite{TheEffectofInflationonEconomicDevelopment}}

    \item Rising prices reduce purchasing power, disproportionately affecting wage earners. \footnotemark[1]

    \item Understanding drivers of inflation is crucial.
    \vspace{1em}
    \item \textbf{We focus on whether geographic coordinates (longitude, latitude) influence inflation rates in Europe.}
  \end{itemize}
\end{frame}

% Data
\begin{frame}{Data}
  \begin{itemize}
    \item Inflation data: \emph{OECD dataset} (29 European countries, Dec. 2005 – Sep. 2024).
    \begin{itemize}    
        \item Removed missing time periods for consistency.
    \end{itemize}
    \item Geographic data: Google’s \href{https://developers.google.com/public-data/docs/canonical/countries_csv}{\emph{countries.csv}}, filtered for European countries.
  \end{itemize}
\end{frame}

% Methodology
\begin{frame}{Methodology}
  \begin{itemize}
    \item Longitude-latitude
    \begin{itemize}
        \item Use centroid-based longitude and latitude for each European country.
    \end{itemize}
    \item Model
    \begin{itemize}
        \item Linear regressions on inflation using longitude, latitude, and both.
    \end{itemize}
    \vspace{1em}
    \item \textbf{Goal: Identify any measurable geographic influence on inflation.}
  \end{itemize}
\end{frame}

% Results Latitude regression
\begin{frame}{Results Latitude regression}
  \centering
  \begin{tabular}{cccc}
      \hline
      {} &  Coefficient &  Std. Error &   P-Value \\
      \hline  
      Intercept &     7.108 &    4.0227 &  0.0885 \\
      latitude  &    -0.0767 &    0.0791 &  0.3413 \\
      \hline
      R-squared     &     0.043\\
      F-Statistic probability &   0.341\\
      \hline
      \vspace{1em}
      \end{tabular}
      
  
  \begin{itemize}
      \item The results of the latitude regression are not statistically significant
  \end{itemize}

\end{frame}

% Results of Longitude Regression

\begin{frame}{Results Longitude Regression}
  \centering
  \begin{tabular}{cccc}
      \hline
      {} &  Coefficient &  Std. Error &   P-Value \\
      \hline
      Intercept &     1.9919 &    0.7193 &  0.0100 \\
      longitude &     0.1061 &    0.0424 &  0.0187 \\
      \hline
      R-squared     &     0.188\\
      F-Statistic probability &   0.0188\\
      \hline
      \vspace{1em}
      \end{tabular}

  \begin{itemize}
      \item The results of the longitude regression are statistically significant at the 5\% significance level
      \item This indicates a positive correlation between the degree of longitude and the mean inflation rate
      \item Simplified this means moving more to the east results in a higher inflation rate
  \end{itemize}
  
\end{frame}

\begin{frame}{Results of combined regression}
  \centering
  \begin{tabular}{cccc}
      \hline
      {} &  Coefficient &  Std. Error &   P-Value \\
      \hline
      Intercept &     4.8368 &    3.8383 &  0.2188 \\
      latitude  &    -0.0556 &    0.0737 &  0.4571 \\
      longitude &     0.1022 &    0.0430 &  0.0253 \\
      \hline
      R-squared     &    0.205\\
      F-Statistic Probability &   0.0503\\
      \hline
      \vspace{1em}
      \end{tabular}
      
  \begin{itemize}
      \item Similar results to the univariate regressions 
      \item Latitude remains statistically insignificant
      \item Longitude remains statistically significant
  \end{itemize}
\end{frame}

% Discussion
\begin{frame}{Discussion}

  \begin{itemize}
      \item Latitude
      \begin{itemize}
          \item There seems to be no north-south effect
      \end{itemize}
      \item Longitude
      \begin{itemize}
          \item Mean inflation rates seem to increase moving eastward
      \end{itemize}
      \item R-squared
      \begin{itemize}
          \item The R-squared of the combined regression is 20.5\% 
          \item This suggests only limited explanatory power of the model
      \end{itemize}
      
      \end{itemize}
  \vspace{1em}

  \begin{itemize}
      \item Limitations
      \begin{itemize}
          \item Longitude data seems to be heavily influenced by Turkey, which lies is one of the most eastern countries and has a high inflation rate
          \item The low R-squared value suggests that more factors should be included to model inflation effectively
      \end{itemize}
  \end{itemize}

\end{frame}

% References
\begin{frame}[allowframebreaks]{References}
\printbibliography
\end{frame}

\end{document}